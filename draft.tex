%!TEX TS-program = xelatex
\documentclass[11pt]{article}

\usepackage[english]{babel}

\usepackage{amsmath,amssymb,amsfonts}
\usepackage[utf8]{inputenc}
\usepackage[T1]{fontenc}
\usepackage{stix2}
\usepackage[scaled]{helvet}
\usepackage[scaled]{inconsolata}

\usepackage{lastpage}

\usepackage{setspace}

\usepackage{ccicons}

\usepackage[hang,flushmargin]{footmisc}

\usepackage{geometry}

\setlength{\parindent}{0pt}
\setlength{\parskip}{6pt plus 2pt minus 1pt}

\usepackage{fancyhdr}
\renewcommand{\headrulewidth}{0pt}\providecommand{\tightlist}{%
  \setlength{\itemsep}{0pt}\setlength{\parskip}{0pt}}

\makeatletter
\newcounter{tableno}
\newenvironment{tablenos:no-prefix-table-caption}{
  \caption@ifcompatibility{}{
    \let\oldthetable\thetable
    \let\oldtheHtable\theHtable
    \renewcommand{\thetable}{tableno:\thetableno}
    \renewcommand{\theHtable}{tableno:\thetableno}
    \stepcounter{tableno}
    \captionsetup{labelformat=empty}
  }
}{
  \caption@ifcompatibility{}{
    \captionsetup{labelformat=default}
    \let\thetable\oldthetable
    \let\theHtable\oldtheHtable
    \addtocounter{table}{-1}
  }
}
\makeatother

\usepackage{array}
\newcommand{\PreserveBackslash}[1]{\let\temp=\\#1\let\\=\temp}
\let\PBS=\PreserveBackslash

\usepackage[breaklinks=true]{hyperref}
\hypersetup{colorlinks,%
citecolor=blue,%
filecolor=blue,%
linkcolor=blue,%
urlcolor=blue}
\usepackage{url}

\usepackage{caption}
\setcounter{secnumdepth}{0}
\usepackage{cleveref}

\usepackage{graphicx}
\makeatletter
\def\maxwidth{\ifdim\Gin@nat@width>\linewidth\linewidth
\else\Gin@nat@width\fi}
\makeatother
\let\Oldincludegraphics\includegraphics
\renewcommand{\includegraphics}[1]{\Oldincludegraphics[width=\maxwidth]{#1}}

\usepackage{longtable}
\usepackage{booktabs}

\usepackage{color}
\usepackage{fancyvrb}
\newcommand{\VerbBar}{|}
\newcommand{\VERB}{\Verb[commandchars=\\\{\}]}
\DefineVerbatimEnvironment{Highlighting}{Verbatim}{commandchars=\\\{\}}
% Add ',fontsize=\small' for more characters per line
\usepackage{framed}
\definecolor{shadecolor}{RGB}{248,248,248}
\newenvironment{Shaded}{\begin{snugshade}}{\end{snugshade}}
\newcommand{\KeywordTok}[1]{\textcolor[rgb]{0.13,0.29,0.53}{\textbf{#1}}}
\newcommand{\DataTypeTok}[1]{\textcolor[rgb]{0.13,0.29,0.53}{#1}}
\newcommand{\DecValTok}[1]{\textcolor[rgb]{0.00,0.00,0.81}{#1}}
\newcommand{\BaseNTok}[1]{\textcolor[rgb]{0.00,0.00,0.81}{#1}}
\newcommand{\FloatTok}[1]{\textcolor[rgb]{0.00,0.00,0.81}{#1}}
\newcommand{\ConstantTok}[1]{\textcolor[rgb]{0.00,0.00,0.00}{#1}}
\newcommand{\CharTok}[1]{\textcolor[rgb]{0.31,0.60,0.02}{#1}}
\newcommand{\SpecialCharTok}[1]{\textcolor[rgb]{0.00,0.00,0.00}{#1}}
\newcommand{\StringTok}[1]{\textcolor[rgb]{0.31,0.60,0.02}{#1}}
\newcommand{\VerbatimStringTok}[1]{\textcolor[rgb]{0.31,0.60,0.02}{#1}}
\newcommand{\SpecialStringTok}[1]{\textcolor[rgb]{0.31,0.60,0.02}{#1}}
\newcommand{\ImportTok}[1]{#1}
\newcommand{\CommentTok}[1]{\textcolor[rgb]{0.56,0.35,0.01}{\textit{#1}}}
\newcommand{\DocumentationTok}[1]{\textcolor[rgb]{0.56,0.35,0.01}{\textbf{\textit{#1}}}}
\newcommand{\AnnotationTok}[1]{\textcolor[rgb]{0.56,0.35,0.01}{\textbf{\textit{#1}}}}
\newcommand{\CommentVarTok}[1]{\textcolor[rgb]{0.56,0.35,0.01}{\textbf{\textit{#1}}}}
\newcommand{\OtherTok}[1]{\textcolor[rgb]{0.56,0.35,0.01}{#1}}
\newcommand{\FunctionTok}[1]{\textcolor[rgb]{0.00,0.00,0.00}{#1}}
\newcommand{\VariableTok}[1]{\textcolor[rgb]{0.00,0.00,0.00}{#1}}
\newcommand{\ControlFlowTok}[1]{\textcolor[rgb]{0.13,0.29,0.53}{\textbf{#1}}}
\newcommand{\OperatorTok}[1]{\textcolor[rgb]{0.81,0.36,0.00}{\textbf{#1}}}
\newcommand{\BuiltInTok}[1]{#1}
\newcommand{\ExtensionTok}[1]{#1}
\newcommand{\PreprocessorTok}[1]{\textcolor[rgb]{0.56,0.35,0.01}{\textit{#1}}}
\newcommand{\AttributeTok}[1]{\textcolor[rgb]{0.77,0.63,0.00}{#1}}
\newcommand{\RegionMarkerTok}[1]{#1}
\newcommand{\InformationTok}[1]{\textcolor[rgb]{0.56,0.35,0.01}{\textbf{\textit{#1}}}}
\newcommand{\WarningTok}[1]{\textcolor[rgb]{0.56,0.35,0.01}{\textbf{\textit{#1}}}}
\newcommand{\AlertTok}[1]{\textcolor[rgb]{0.94,0.16,0.16}{#1}}
\newcommand{\ErrorTok}[1]{\textcolor[rgb]{0.64,0.00,0.00}{\textbf{#1}}}
\newcommand{\NormalTok}[1]{#1}

\newlength{\cslhangindent}
\setlength{\cslhangindent}{1.5em}
\newlength{\csllabelwidth}
\setlength{\csllabelwidth}{3em}
\newenvironment{CSLReferences}[3] % #1 hanging-ident, #2 entry spacing
 {% don't indent paragraphs
  \setlength{\parindent}{0pt}
  % turn on hanging indent if param 1 is 1
  \ifodd #1 \everypar{\setlength{\hangindent}{\cslhangindent}}\ignorespaces\fi
  % set entry spacing
  \ifnum #2 > 0
  \setlength{\parskip}{#2\baselineskip}
  \fi
 }%
 {}
\usepackage{calc} % for \widthof, \maxof
\newcommand{\CSLBlock}[1]{#1\hfill\break}
\newcommand{\CSLLeftMargin}[1]{\parbox[t]{\maxof{\widthof{#1}}{\csllabelwidth}}{#1}}
\newcommand{\CSLRightInline}[1]{\parbox[t]{\linewidth}{#1}}
\newcommand{\CSLIndent}[1]{\hspace{\cslhangindent}#1}\geometry{verbose,letterpaper,tmargin=2.2cm,bmargin=2.2cm,lmargin=2.2cm,rmargin=2.2cm}

\usepackage{lineno}
\usepackage[nolists,noheads]{endfloat}

\pagestyle{plain}

\tolerance=1
\emergencystretch=\maxdimen
\hyphenpenalty=10000
\hbadness=10000

\doublespacing

\fancypagestyle{normal}
{
  \fancyhf{}
  \fancyfoot[R]{\footnotesize\sffamily\thepage\ of \pageref*{LastPage}}
}
\begin{document}
\raggedright
\thispagestyle{empty}
{\Large\bfseries\sffamily test d'un template basé sur du markdown \&
github pour soumettre un article à revue}
\vskip 5em

%
\href{https://orcid.org/0000-0000-0000-0000}{choupinou\,Brocoli}%
%
\,\textsuperscript{1,2,‡}\quad %
Peregrin\,Took%
%
\,\textsuperscript{3,4}\quad %
Merriadoc\,Brandybuck%
%
\,\textsuperscript{5,4,‡}

\textsuperscript{1}\,INRAE\quad \textsuperscript{2}\,Revue\quad \textsuperscript{3}\,Inn
of the Prancing Pony\quad \textsuperscript{4}\,Fellowship of the
Ring\quad \textsuperscript{5}\,Green Dragon Inn

\textsuperscript{‡}\,These authors contributed equally to the work\\

\textbf{Correspondance to:}\\
choupinou Brocoli --- \texttt{brocoli.choupinou@inrae.fr}\\

\vfill
This work is released by its authors under a CC-BY 4.0 license\hfill\ccby\\
Last revision: \emph{\today}

\clearpage
\thispagestyle{empty}

\vfill


        {\bfseries Purpose:}\,This template provides a series of scripts
to render a markdown document into an interactive website and a series
of PDFs.\\%
        {\bfseries Motivation:}\,It makes collaborating on text with
GitHub easier, and means that we never need to think about the
output.\\%
        {\bfseries Data availability statement:}\,The data that support
the findings of this study are openly available in the
gardenRepository.\\%
    
\vfill

\clearpage
\linenumbers
\pagestyle{normal}

le contenu de mon preprint est ici.

\hypertarget{the-metadata-file}{%
\section{The metadata file}\label{the-metadata-file}}

\hypertarget{general-information}{%
\subsection{General information}\label{general-information}}

The title is a field in the \texttt{metadata.json}:

\begin{Shaded}
\begin{Highlighting}[]
\FunctionTok{\{}
    \DataTypeTok{"title"}\FunctionTok{:} \StringTok{" test d\textquotesingle{}un fictional{-}Broccoli preprint"}
\FunctionTok{\}}
\end{Highlighting}
\end{Shaded}

\hypertarget{authorship}{%
\subsection{Authorship}\label{authorship}}

Authors are listed as objects in the \texttt{authors} block. Each author
is specified as follows:

\begin{Shaded}
\begin{Highlighting}[]
\FunctionTok{\{}
      \DataTypeTok{"familyname"}\FunctionTok{:} \StringTok{"PEIFFER"}\FunctionTok{,}
      \DataTypeTok{"givennames"}\FunctionTok{:} \StringTok{"Marianne"}\FunctionTok{,}
      \DataTypeTok{"email"}\FunctionTok{:} \StringTok{"marianne.peifferb@inrae.fr"}\FunctionTok{,}
      \DataTypeTok{"orcid"}\FunctionTok{:} \StringTok{"0000{-}0000{-}0000{-}0001"}\FunctionTok{,}
      \DataTypeTok{"affiliations"}\FunctionTok{:} \OtherTok{[}
        \StringTok{"INRAE"}\OtherTok{,}
        \StringTok{"Affiliation 2"}
      \OtherTok{]}\FunctionTok{,}
      \DataTypeTok{"status"}\FunctionTok{:} \OtherTok{[}\StringTok{"corresponding"}\OtherTok{,} \StringTok{"equal"}\OtherTok{]}
    \FunctionTok{\}}
\end{Highlighting}
\end{Shaded}

The \texttt{email} field is recommended for all authors. The
\texttt{status} field is only useful for the corresponding author, and
to denote equal contributions. These informations are rendered on the
initial page. If an \texttt{orcid} is given, it will be linked on the
HTML and PDF versions.

Note that there is \emph{no need} to number the affiliations - a small
python script will take care of this automatically.

\hypertarget{abstract}{%
\subsection{Abstract}\label{abstract}}

\begin{Shaded}
\begin{Highlighting}[]
\ErrorTok{"abstract":} \OtherTok{[}
    \StringTok{"Point 1"}\OtherTok{,} \ErrorTok{le} \DecValTok{1} \ErrorTok{er} \ErrorTok{élément} \ErrorTok{de} \ErrorTok{mon} \ErrorTok{résumé}
    \StringTok{"Point 2"} \ErrorTok{et} \ErrorTok{le} \DecValTok{2} \ErrorTok{ème} \ErrorTok{élément} \ErrorTok{de} \ErrorTok{mon} \ErrorTok{résumé}
\OtherTok{]}
\end{Highlighting}
\end{Shaded}

\hypertarget{references}{%
\section{References}\label{references}}

\begin{verbatim}
[auth:fold]
[year]
[title:fold:nopunctordash:skipwords:lower:select=1,1:substring=1,3:capitalize]
[title:fold:nopunctordash:skipwords:lower:select=2,2:substring=1,3:capitalize]
\end{verbatim}

\hypertarget{figures-tables-and-other-floats}{%
\section{Figures, Tables, and other
floats}\label{figures-tables-and-other-floats}}

Note that you can wrap the text of legends for both figures and tables.
This avoids the issue of having very long lines.

\hypertarget{mathematics}{%
\subsection{Mathematics}\label{mathematics}}

The following equation

\begin{equation}\protect\hypertarget{eq:eq1}{}{J'(p) = \frac{1}{\text{log}(S)}\times\left(-\sum p \times \text{log}(p)\right)}\label{eq:eq1}\end{equation}

is produced using

\begin{Shaded}
\begin{Highlighting}[]
\SpecialStringTok{$$J\textquotesingle{}(p) = }\SpecialCharTok{\textbackslash{}frac}\SpecialStringTok{\{1\}\{}\SpecialCharTok{\textbackslash{}text}\NormalTok{\{log\}}\SpecialStringTok{(S)\}}\SpecialCharTok{\textbackslash{}times}\SpecialStringTok{ ... $$}\NormalTok{ \{\#eq:eq1\}}
\end{Highlighting}
\end{Shaded}

and can be referenced using \texttt{@eq:eq1}, which will result in
eq.~\ref{eq:eq1}. Note that because we use \texttt{pandoc-crossref}, the
label ``eq.'' will be generated automatically.

\hypertarget{tables}{%
\subsection{Tables}\label{tables}}

Table legends go on the line after the table itself. To generate a
reference to the table, use \texttt{\{\#tbl:id\}} -- then, in the text,
you can use \texttt{\{@tbl:id\}} to refer to the table. For example, the
table below is tbl.~\ref{tbl:id}. You can remove the \emph{table} in
front by using \texttt{!@tbl:id}, or force it to be capitalized with
\texttt{\textbackslash{}*tbl:id}.

\hypertarget{tbl:id}{}
\begin{longtable}[]{@{}rrrrl@{}}
\caption{\label{tbl:id}Tableau du TEST BROCCOLI, qui n'a pas de doi
\texttt{id} -- we can refer to it using \texttt{\{@tbl:id\}}. Note that
even if the table legend is written below the table itself, it will
appear on top in the PDF document.}\tabularnewline
\toprule
\begin{minipage}[b]{0.20\columnwidth}\raggedleft
Sepal.Length du BROCCOLI\strut
\end{minipage} & \begin{minipage}[b]{0.19\columnwidth}\raggedleft
Sepal.Widthdu BROCCOLI\strut
\end{minipage} & \begin{minipage}[b]{0.20\columnwidth}\raggedleft
Petal.Length du BROCCOLI\strut
\end{minipage} & \begin{minipage}[b]{0.19\columnwidth}\raggedleft
Petal.Width du BROCCOLI\strut
\end{minipage} & \begin{minipage}[b]{0.07\columnwidth}\raggedright
BROCCOLI\strut
\end{minipage}\tabularnewline
\midrule
\endfirsthead
\toprule
\begin{minipage}[b]{0.20\columnwidth}\raggedleft
Sepal.Length du BROCCOLI\strut
\end{minipage} & \begin{minipage}[b]{0.19\columnwidth}\raggedleft
Sepal.Widthdu BROCCOLI\strut
\end{minipage} & \begin{minipage}[b]{0.20\columnwidth}\raggedleft
Petal.Length du BROCCOLI\strut
\end{minipage} & \begin{minipage}[b]{0.19\columnwidth}\raggedleft
Petal.Width du BROCCOLI\strut
\end{minipage} & \begin{minipage}[b]{0.07\columnwidth}\raggedright
BROCCOLI\strut
\end{minipage}\tabularnewline
\midrule
\endhead
\begin{minipage}[t]{0.20\columnwidth}\raggedleft
5.1\strut
\end{minipage} & \begin{minipage}[t]{0.19\columnwidth}\raggedleft
3.5\strut
\end{minipage} & \begin{minipage}[t]{0.20\columnwidth}\raggedleft
1.4\strut
\end{minipage} & \begin{minipage}[t]{0.19\columnwidth}\raggedleft
0.2\strut
\end{minipage} & \begin{minipage}[t]{0.07\columnwidth}\raggedright
oui\strut
\end{minipage}\tabularnewline
\begin{minipage}[t]{0.20\columnwidth}\raggedleft
5.0\strut
\end{minipage} & \begin{minipage}[t]{0.19\columnwidth}\raggedleft
3.6\strut
\end{minipage} & \begin{minipage}[t]{0.20\columnwidth}\raggedleft
1.4\strut
\end{minipage} & \begin{minipage}[t]{0.19\columnwidth}\raggedleft
0.2\strut
\end{minipage} & \begin{minipage}[t]{0.07\columnwidth}\raggedright
oui\strut
\end{minipage}\tabularnewline
\begin{minipage}[t]{0.20\columnwidth}\raggedleft
5.4\strut
\end{minipage} & \begin{minipage}[t]{0.19\columnwidth}\raggedleft
3.9\strut
\end{minipage} & \begin{minipage}[t]{0.20\columnwidth}\raggedleft
1.7\strut
\end{minipage} & \begin{minipage}[t]{0.19\columnwidth}\raggedleft
0.4\strut
\end{minipage} & \begin{minipage}[t]{0.07\columnwidth}\raggedright
oui\strut
\end{minipage}\tabularnewline
\bottomrule
\end{longtable}

\hypertarget{figures}{%
\section{Figures}\label{figures}}

\begin{verbatim}
![Image libre de droit de brocoli trouvée sur https://www.publicdomainpictures.net](figures/image.png){#fig:figure}
\end{verbatim}

\begin{figure}
\hypertarget{fig:figure}{%
\centering
\includegraphics{figures/image2.png}
\caption{Image libre de droit de brocoli trouvée sur
https://www.publicdomainpictures.net}\label{fig:figure}
}
\end{figure}

We can now use \texttt{@fig:figure} to refer to fig.~\ref{fig:figure}.

\hypertarget{example-text}{%
\section{Example text}\label{example-text}}

Le brocoli est une variété de chou originaire de Sicile. Il fut
sélectionné par les Romains à partir du chou sauvage
\begin{equation}\protect\hypertarget{eq:cstar}{}{Co^\star=\frac{L-c_m}{T\times B-c_m} \,.}\label{eq:cstar}\end{equation}

\hypertarget{brassica-oleracea}{%
\subsection{Brassica oleracea}\label{brassica-oleracea}}

un chou est un chou

\hypertarget{brassica-oleracea-var.-italica}{%
\subsection{Brassica oleracea var.
italica}\label{brassica-oleracea-var.-italica}}

une variété italienne existe.

\hypertarget{references-1}{%
\section{References}\label{references-1}}

\end{document}
